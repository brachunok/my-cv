
\documentclass[11pt,article,oneside]{memoir}
\usepackage[english]{babel}
\usepackage{titlesec}
\usepackage{setspace}

% margins
\usepackage[left=1.5in, right=1.75in, top=1in, bottom=0.75in]{geometry}% adjust the page margins

\RequirePackage{xcolor}
\definecolor{CornellRed}{HTML}{b31b1b} % official Cornell red color -- use for text etc
\definecolor{CornellGray}{HTML}{222222} % official Cornell gray -- use for Section
\definecolor{LightGray}{HTML}{F7F7F7} % background color -- do not use for anything else
\definecolor{BrickRed}{HTML}{C04829} % backup color
\definecolor{GrassGreen}{HTML}{359245}
\definecolor{WarmYellow}{HTML}{E9A139}

%% Choose fonts for use with xelatex
\usepackage{fontspec}
\setmainfont[Path=fonts/,
Mapping={tex-text},
Numbers={OldStyle},
Ligatures={Common},
UprightFont=*-Regular,
BoldFont=*-Bold,
ItalicFont=*-Italic,
BoldItalicFont=*-BoldItalic
]{CrimsonText}
\setsansfont[Path=fonts/,
Mapping=tex-text,
Colour=b31b1b,
UprightFont=*-Regular,
BoldFont=*-Bold,
ItalicFont=*-Italic]{Lato}
\setmonofont[Path=fonts/,
Mapping=tex-text,
Scale=0.9,
UprightFont=*-Regular]{Inconsolata}

%%------------------------------------------------------------------------
%% Section styling
%%------------------------------------------------------------------------

%% This includes a fudge from the following link in order to get the
%% subsection to align with a section
%% http://tex.stackexchange.com/questions/19200/titlesec-remove-space-after-empty-margin-section

\makeatletter
\newif\ifaftersec\aftersecfalse

\newcommand\setsubskip{%
    \global\aftersectrue
    \everypar{%
        \global\aftersecfalse
        \if@noskipsec
            \global\@noskipsecfalse
            \clubpenalty\@M
            \hskip-\parindent
            \begingroup
                \@svsechd\unskip{\hspace{\@tempskipb}}%
            \endgroup
        \else
            \clubpenalty\@clubpenalty\everypar{}%
        \fi}}

\newcommand\subskip{%
  \ifaftersec
     \removelastskip%
     \vspace{-\baselineskip}
  \fi
  \global\aftersecfalse}

% Section styling
\titleformat{\section}[leftmargin]{\raggedleft\sffamily\bfseries\footnotesize}{}{0pt}{}[\setsubskip]
\titlespacing*{\section}{2cm}{1ex}{0.25in}

% Subsection styling
\titleformat{\subsection}{\subskip\itshape}{}{0pt}{}[]
\titlespacing*{\subsection}{0em}{2.5ex}{1ex}

\raggedbottom
\makeatother

% useful macro
\newcommand{\dotspace}{\quad}

% better lists
\usepackage{enumitem}
\setlist[itemize, 1]{nosep, noitemsep, labelsep=-5pt, leftmargin=*, after=\vspace{\baselineskip}, itemindent=-15pt, labelindent=5pt}

% Use BibLaTeX
\usepackage[autostyle=true, autopunct=true, english=american]{csquotes}
\usepackage[
	backend=biber,
	style=publist,
	plauthorfirstinit=true,
	plauthorhandling=highlight,
	plnumbered=false,
	labeldateparts=false,
	%linktitleall=true,
	boldyear=false,
	marginyear=false,
	doi=true,
	url=false,
	isbn=false,
	maxbibnames=99,
	maxcitenames=99,
  defernumbers,
  giveninits,
	date=year
]{biblatex}

\setlength{\bibhang}{15pt}

% add bib resources
((* set bibpath = data.paths.bib_path *))
((* for i in data.publications *))
\addbibresource{(((bibpath)))/(((i.file)))}
((* endfor *))

% include everything in bibliography
\nocite{*}

% bold my name
\plauthorname[Vivek]{Srikrishnan}

% make modifications to standard styles
\DeclareFieldFormat{title}{\mkbibquote{#1}}
\DeclareFieldFormat{url}{\url{#1}}
\DeclareFieldFormat[article]{pages}{#1}
\DeclareFieldFormat[inproceedings]{pages}{\lowercase{pp.}#1}
\DeclareFieldFormat[incollection]{pages}{\lowercase{pp.}#1}
\DeclareFieldFormat[article]{volume}{\mkbibbold{#1}}
\DeclareFieldFormat[article]{number}{\mkbibparens{#1}}
\DeclareFieldFormat[article]{url}{}
\DeclareFieldFormat{shorthandwidth}{#1}
\DeclareFieldFormat{extrayear}{}

% Remove In: for an article.
\renewbibmacro{in:}{%
  \ifentrytype{article}{}{%
  \printtext{\bibstring{in}\intitlepunct}}}

% Format sub-bibliography headings as subsections
\defbibheading{subbibliography}[\refname]{\subsection*{#1}}

%% other stuff
% icons for the contact info
\usepackage{fontawesome5}
\usepackage{academicons}

% macros and environments
\newcommand{\Header}[1]{\\\multicolumn{3}{L{7in}}{\color{BrickRed}\Large{\spacedlowsmallcaps{#1}}}\\}
\newcommand{\Publication}[1]{\fullcite{#1} & \citeyear{#1}}
\newcommand{\CVEntry}[3]{#1 & #2 & #3}
\newenvironment{cv}
{
	\noindent\begin{longtable}{ >{\em}R{1in} L{5in} >{\em}R{0.875in} }
}{
    \end{longtable}
}

\usepackage{fancyhdr}
\usepackage[yyyymmdd,hhmmss]{datetime}
\pagestyle{fancy}
\rfoot{Current as of \today}
\cfoot{}
\lfoot{Page \thepage}

\renewcommand{\headrulewidth}{0pt}
\fancyhead{}

% have to load this last
\usepackage{hyperref} % Required for adding links	and customizing them
\hypersetup{colorlinks, breaklinks, urlcolor=CornellRed, linkcolor=CornellRed, pdfnewwindow=true}

%---------------------------------------------------------------------

\begin{document}

%% Name and contact block

\begin{minipage}[t]{2.8in}
 \flushright {\footnotesize
 \href{(((data.person.affiliation_link)))}{(((data.person.affiliation | escape_tex)))}\\ \vspace{0in}
 (((data.person.address.line1)))\\ \vspace{0in} (((data.person.address.line2))) \\ \vspace{-0.04in} (((data.person.address.postcode))), (((data.person.address.country)))}
 \end{minipage}
 \begin{minipage}[t]{2.5in}
   \flushright {\footnotesize  \texttt{\href{mailto:(((data.person.email)))}{(((data.person.email)))}} \, \faEnvelope} \\ \vspace{-0.21in}
   \flushright {\footnotesize  \texttt{\href{(((data.person.web)))}{(((data.person.web)))}} \, \faGlobe} \\ \vspace{-0.195in}
   \flushright {\footnotesize
   \texttt{\href{https://github.com/(((data.person.github)))}{(((data.person.github)))}} \, \faGithub} \\ \vspace{-0.2in}
   \flushright {\footnotesize
   \texttt{\href{https://scholar.google.com/citations?user=(((data.person.googlescholar)))}{(((data.person.first_name))) (((data.person.last_name)))}} \, \aiGoogleScholar}

 \end{minipage}

 \bigskip

 %% Name
 \noindent{\LARGE\bfseries \textcolor{CornellGray}{(((data.person.first_name))) (((data.person.last_name)))}}
 \reversemarginpar

\bigskip

%% Research Interests
((* set sect = data.sections | select_by_attr_name("title", "Research Interests") *))
\section{(((sect.title | lower)))}

\mbox{}\vspace{-\dimexpr\baselineskip\relax}

\begin{itemize}[label={}]
((* for i in sect.entries.interests *))
\item (((i.name)))
((* endfor *))
\end{itemize}

%% Appointments
((* set sect = data.sections | select_by_attr_name("title", "Appointments") *))
\section{(((sect.title | lower)))}

\mbox{}\vspace{-\dimexpr\baselineskip\relax}

\begin{itemize}[label={}]
((* for i in sect.entries.appointments | sort_by_attr(["start"], reverse=True) *))
\item (((i.title))), (((i.dept | escape_tex))), (((i.univ))), (((i.start)))--(((i.end)))
((* endfor *))
\end{itemize}

%% Education
((* set sect = data.sections | select_by_attr_name("title", "Education")*))
\section{(((sect.title | lower)))}

\mbox{}\vspace{-\dimexpr\baselineskip\relax}

\begin{itemize}[label={}]
((* for i in sect.entries.degrees | sort_by_attr(["year"], reverse=True) *))
\item (((i.degree))), (((i.subject | escape_tex))), (((i.school))), (((i.year)))
((* endfor *))
\end{itemize}

%% Publications
\section{publications}

\printbibliography[type=article, title={Peer-Reviewed Journal Articles}, heading=subbibliography]

\printbibliography[type=misc, title={Articles Under Review or Forthcoming}, heading=subbibliography, keyword=unpublished]

\printbibliography[type=inproceedings, title={Conference Proceedings}, heading=subbibliography]

\printbibliography[type=misc, title={Conference Presentations}, heading=subbibliography, keyword=talk]

\printbibliography[type=misc, title={Conference Posters}, heading=subbibliography, keyword=poster]

%% Invited Talks
((* set sect = data.sections | select_by_attr_name("title", "Invited Talks") *))
\section{(((sect.title | lower)))}

\mbox{}\vspace{-\dimexpr\baselineskip\relax}

\begin{itemize}[label={}]
((* for i in sect.entries.talks | sort_by_attr(["year"], reverse=True) *))
\item \enquote{(((i.title | escape_tex)))}, ((* if i.event *))(((i.event | escape_tex))), ((* endif *)) (((i.org))). (((i.location))). (((i.month)))((* if i.month != "May" *)).((* endif *)) (((i.year)))((* if i.coauthor *)) (with (((i.coauthor))))((* endif *)).
((* endfor *))
\end{itemize}

%% Workshops
((* set sect = data.sections | select_by_attr_name("title", "Workshops Organized") *))
\section{(((sect.title | lower)))}

\mbox{}\vspace{-\dimexpr\baselineskip\relax}

\begin{itemize}[label={}]
((* for i in sect.entries.workshops | sort_by_attr(["year"], reverse=True) *))
\item (((i.title | escape_tex))). (((i.location))). (((i.month)))((* if i.month != "May" *)).((* endif *)) (((i.year))).
((* endfor *))
\end{itemize}

%% Grants and Contracts
((* set sect = data.sections | select_by_attr_name("title", "Grants and Contracts") *))
\section{(((sect.title | lower)))}

\mbox{}\vspace{-\dimexpr\baselineskip\relax}

\begin{itemize}[label={}]
((* for i in sect.entries.grants | sort_by_attr(["start"], reverse=True) *))
\item (((i.role))), \enquote{(((i.title | escape_tex)))}. ((* if i.PI *)) PI: (((i.PI))).((* endif *)) (((i.funder))), \$(((i.amount)))((* if i.total *)) (total: \$(((i.total))))((* endif *)). (((i.start)))--(((i.end))).
((* endfor *))
\end{itemize}

%% Networks and Projects
((* set sect = data.sections | select_by_attr_name("title", "Networks and Projects") *))
\section{(((sect.title | lower)))}

\mbox{}\vspace{-\dimexpr\baselineskip\relax}

\begin{itemize}[label={}]
((* for i in sect.entries.projects | sort_by_attr(["start"], reverse=True) *))
\item (((i.title | escape_tex))). (((i.role))). PI: (((i.PI))). (((i.funder))). (((i.start)))--(((i.end))).
((* endfor *))
\end{itemize}

%% Teaching
((* set sect = data.sections | select_by_attr_name("title", "Teaching") *))
\section{(((sect.title | lower)))}

\mbox{}\vspace{-\dimexpr\baselineskip\relax}

\vspace{\baselineskip}

((* set courses = sect.entries.classes *))
((* for u, i in courses | groupby("university")  *))
\subsection{(((u)))}
\begin{itemize}[label={}]
((* for x in i | sort_first_year("year", reverse=True) *))
\item (((x.title | escape_tex))). (((x.role))). (((x. semester))) (((x.year))).
((* endfor *))
\end{itemize}
((* endfor *))

%% Advising
((* set sect = data.sections | select_by_attr_name("title", "Advising") *))
\section{(((sect.title | lower)))}

\mbox{}\vspace{-\dimexpr\baselineskip\relax}

\vspace{\baselineskip}
((* set advisees = sect.entries.advising *))
((* for u, i in advisees | groupby("university") *))
\subsection{(((u)))}
\begin{itemize}[label={}]
((* for x in i | sort_by_attr(["end", "start"], reverse=True) *))
\item (((x.name))), (((x.degree))) (((x.field))). (((x.role))). (((x.start)))--(((x.end))).
((* endfor *))
\end{itemize}
((* endfor *))

%% Awards
((* set sect = data.sections | select_by_attr_name("title", "Awards") *))
\section{(((sect.title | lower)))}

\mbox{}\vspace{-\dimexpr\baselineskip\relax}

\begin{itemize}[label={}]
((* for i in sect.entries.awards | sort_by_attr(["year"], reverse=True) *))
\item (((i.title))), (((i.org))). (((i.year)))
((* endfor *))
\end{itemize}

%% Outreach
((* set sect = data.sections | select_by_attr_name("title", "Outreach") *))
\section{(((sect.title | lower)))}

\mbox{}\vspace{-\dimexpr\baselineskip\relax}

\begin{itemize}[label={}]
((* for i in sect.entries.outreach | sort_first_year("year", reverse=True) *))
\item (((i.activity))), (((i.org))). (((i.year))).
((* endfor *))
\end{itemize}

%% Service
((* set sect = data.sections | select_by_attr_name("title", "Service") *))
\section{(((sect.title | lower)))}

\mbox{}\vspace{-\dimexpr\baselineskip\relax}

\begin{itemize}[label={}]
((* for i in sect.entries.service *))
\item (((i.title | escape_tex))),((* if i.journal *)) \emph{(((i.journal)))}.((* endif *)) ((* if i.org *))(((i.org | escape_tex))).((* endif *)) (((i.start)))((* if i.end *))--(((i.end)))((* endif *)).
((* endfor *))
\end{itemize}

%% Professional Experience
((* set sect = data.sections | select_by_attr_name("title", "Professional Experience") *))
\section{(((sect.title | lower)))}

\mbox{}\vspace{-\dimexpr\baselineskip\relax}

\begin{itemize}[label={}]
((* for i in sect.entries.experience | sort_by_attr(["end", "start"], reverse=True) *))
\item (((i.title))), (((i.org)))). (((i.start)))--(((i.end)))
((* endfor *))
\end{itemize}



\end{document}
